%----------------------------------------------------------------------------------------
%	SKILLS & COMPETENCIES
%----------------------------------------------------------------------------------------

\section{Skills}

\begin{itemize}
    % CATEGORY 1: Programming Languages and Frameworks - Using Step 14 ordering
    % PRIORITY 1: Core technical foundation - programming languages are fundamental
    \item \small{\titlecap{Programming Languages \& Frameworks:}}\footnotesize{ \titlecap{\LanguagesOrderSkills}}
    % IMPROVEMENT: No improvements needed - references macro that will be sorted separately

    % CATEGORY 2: System Architecture and Design - Using Step 14 ordering
    % PRIORITY 1: System architecture - critical for backend/distributed systems roles
    \item \small{\titlecap{System Architecture \& Design:}}\footnotesize{ \titlecap{
        Distributed Systems Design and Architecture, Microservices Patterns
        % Architecture,
        High-Availability
        % System
        % Design,
        Solutions,
        % Scalable System Design, \BackEnd{} Architecture, High-throughput System Design,
        REST and gRPC API
        % Design
        Patterns, Horizontal Scalability, Service Isolation, CAP Theorem, Eventual Consistency, Consensus Algorithms, Multi-tenancy, User Identity Systems, Friend Graph Systems, Architecture
        % Design
        Reviews, High-Performance Computing (HPC)
    }}
    % IMPROVEMENT: "User Identity Services" → "User Identity Systems" (more precise terminology)
    % SORTING: Distributed systems concepts first, then architecture patterns, then specific implementations

    % CATEGORY 3: Data Engineering and Analytics - Using Step 14 ordering
    % PRIORITY 1: Data engineering - directly relevant to data engineering roles
    \item \small{\titlecap{Data Engineering \& Analytics:}}\footnotesize{ \titlecap{
        % Data
        Pipeline Design and Development, Large-Scale
        % Data
        Processing,
        % /Multi-Petabyte Data Processing,
        Apache Spark, Stream
        % ing Data
        Processing, Real-time Analytics, ETL/ELT
        Workflows,
        % Processes, Data
        Modeling, Distributed Transaction Models, High-Performance
        % Data
        Delivery, Google Cloud Dataflow, Apache Beam,
        % Data
        Governance and Quality Assurance,
        % Data
         Partitioning Strategies
        % , BI Dashboards, PowerBI
    }}
    % IMPROVEMENTS: 
    % "Data Pipelines Design and Development" → "Data Pipeline Design and Development" (more standard terminology)
    % "Large Datasets Processing" → "Large-Scale Data Processing" (more professional terminology)
    % "ETL" → "ETL/ELT Processes" (includes modern ELT patterns)
    % "Streaming Data" → "Streaming Data Processing" (more specific)
    % "Dataflow" → "Google Cloud Dataflow" (more specific)
    % "Data Governance and Quality" → "Data Governance and Quality Assurance" (more comprehensive)
    % SORTING: Core data pipeline skills first, then processing technologies, then governance and visualization

    % CATEGORY 4: Data Storage and Databases - Using Step 14 ordering
    % PRIORITY 1: Data storage - essential for backend and data engineering
    \item \small{\titlecap{Data Storage \& Databases:}}\footnotesize{ \titlecap{
        Distributed Storage Solutions (HDFS, Parquet, ORC), SQL Databases, NoSQL Databases (BigTable, Azure Cosmos DB), Cloud Storage (Google Cloud Storage, Azure Data Lake, Google Filestore, BigQuery, Google Pub/Sub), Data Consistency Controls, Persistence Layer Design, Tiered Storage Management, In-memory Caching Strategies
    }}
    % IMPROVEMENTS:
    % "Azure Key-Value Data Store" → "Azure Cosmos DB" (more specific and recognizable)
    % "Filestore" → "Google Filestore" (more specific)
    % "Pub/Sub" → "Google Pub/Sub" (more specific)
    % SORTING: Distributed storage first, then database types, then cloud storage, then optimization strategies

    % CATEGORY 5: Cloud Technologies and Platforms - Using Step 14 ordering
    % PRIORITY 1: Cloud technologies - essential for modern backend/data engineering
    \item \small{\titlecap{Cloud Technologies \& Platforms:}}\footnotesize{ \titlecap{
        Container Orchestration (Kubernetes and Docker), Azure Cloud Services, Google Cloud Platform (GCP) Services, Message Brokers (Apache Kafka, Azure Service Bus), Cloud-Native Applications, Infrastructure-as-Code (IaC), Auto-scaling Infrastructure, Self-healing System Design, Traffic Management and Load Balancing
    }}
    % IMPROVEMENTS:
    % "Cloud-Native Services" → "Cloud-Native Applications" (more precise)
    % "Azure Services" → "Azure Cloud Services" (more specific)
    % "GCP Services" → "Google Cloud Platform (GCP) Services" (expanded for clarity)
    % "Traffic Management" → "Traffic Management and Load Balancing" (more comprehensive)
    % SORTING: Container orchestration first, then cloud platforms, then messaging, then infrastructure patterns

    % CATEGORY 6: Reliability Engineering and Operations - Using Step 14 ordering
    % PRIORITY 1: Reliability engineering - critical for senior backend roles
    \item \small{\titlecap{Reliability Engineering \& Operations:}}\footnotesize{ \titlecap{
        High-Availability Systems, Scalability Engineering, Service
        % System
        Reliability, Monitoring and Observability, Performance Profiling and Tuning,
        % Optimization, System
        Infrastructure Optimization, Bottleneck Identification and Resolution, SLO/SLI Management, Incident Response, Root Cause Analysis (Fault Tree Analysis, 5 Whys), Postmortem Processes, Resilience Patterns (Circuit Breaking, Bulkhead Patterns, Rate Limiting), Metrics Collection and Analytics,
        % Analysis,
        Alerting Frameworks,
        % Systems,
        Centralized Logging
        % , Security Engineering, Operational Excellence - Too generic for the resume
    }}
    % IMPROVEMENTS:
    % "Reliability Engineering and Ops" → "Reliability Engineering and Operations" (more professional)
    % "High-Availability" → "High-Availability Systems" (more specific)
    % "Scalability" → "Scalability Engineering" (more specific)
    % "Reliability" → "System Reliability" (more specific)
    % "Performance Profiling" → "Performance Profiling and Optimization" (more comprehensive)
    % "Observability" → "Monitoring and Observability" (maintains order but adds clarity)
    % "Logging" → "Centralized Logging" (more specific)
    % "Security" → "Security Engineering" (more specific)
    % "Optimization" → "System Optimization" (more specific)
    % SORTING: Core reliability concepts first, then monitoring/optimization, then incident management, then patterns and practices

    % CATEGORY 7: Software Engineering - Using Step 14 ordering
    % PRIORITY 2: Software engineering practices - important foundation but less specific
    \item \small{\titlecap{Software Engineering:}}\footnotesize{ \titlecap{
        Software Design Patterns,
        % \BackEnd{} Architecture,
        Algorithms and Data Structures,
        % Performance Optimization,
        Concurrency and Parallelism,
        % Test Automation and 
        Test-driven Development (TDD), Behavior-driven Development (BDD), CI/CD Pipelines, Code Quality \& Static Analysis, DevOps Practices, Agile Methodologies (XP and Scrum), Software Development Life Cycle (SDLC),
        % A/B Testing
        % , Technical Debt Prioritization, MSBuild
        Bazel Build System
    }}
    % IMPROVEMENTS:
    % "Software Design" → "Software Design Patterns" (more specific)
    % "CI/CD" → "CI/CD Pipelines" (more specific)
    % "DevOps" → "DevOps Practices" (more specific)
    % "Agile (XP and Scrum)" → "Agile Methodologies (XP and Scrum)" (more professional)
    % "SDLC" → "Software Development Life Cycle (SDLC)" (expanded for clarity)
    % "Blaze" → "Bazel Build System" (more recognizable - Blaze is Google's internal version)
    % SORTING: Core engineering concepts first, then testing practices, then development processes, then tools

    % CATEGORY 8: Technical Leadership and Collaboration - Using Step 14 ordering
    % PRIORITY 2: Technical leadership - valuable for senior roles but less technical depth
    \item \small{\titlecap{Technical Leadership \& Collaboration:}}\footnotesize{ \titlecap{
        Engineering
        % Technical
        Leadership, Mentorship and Coaching,
        % Technical
        Trade-off Analysis,
        % Technical
        Roadmap Planning,
        % Cross-team/
        Cross-functional Collaboration,
        % /Teamwork and Interdisciplinary Work,
        Distributed Teams Coordination,
        % Collaborative Problem-Solving,
        Stakeholder Communication, Requirements Gathering and Evaluation,
        % Product Requirements Understanding, Technical
        Documentation and Standards, Knowledge Sharing,
        % Constructive Feedback, Technical
        Adaptability
        % Continuous Learning Aptitude, Growth Mindset
    }}
    % IMPROVEMENTS:
    % "Leadership" → "Technical Leadership" (more specific)
    % "High Aptitude to Learn" → "Continuous Learning Aptitude" (more professional phrasing)
    % SORTING: Leadership skills first, then collaboration skills, then communication and documentation, then personal attributes

    % CATEGORY 9: AI/ML and Data Science
    % PRIORITY 3: AI/ML - relevant for AI companies but not core backend/data engineering
    % \item \small{\titlecap{AI/ML \& Data Science:}}\footnotesize{ \titlecap{AI/ML Fundamentals, Data Ingestion and Preparation, Statistical Analysis, LLM Prompt Engineering}}
    % IMPROVEMENT: "Prompt Engineering" → "LLM Prompt Engineering" (more specific)
    % SORTING: Fundamentals first, then data preparation, then analysis, then specialized skills

    % CATEGORY 10: Languages - Using Step 14 ordering
    % PRIORITY 4: Languages - important but not technical skill
    % \item \small{\titlecap{Languages:}}\footnotesize{ \titlecap{English (Fluent), Portuguese (Native)}}
    % IMPROVEMENT: No improvements needed
    % SORTING: Professional language first, then native language
\end{itemize}